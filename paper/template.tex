\documentclass[10pt, conference, letterpaper]{IEEEtran}

\usepackage{algorithm}
\usepackage{algorithmicx}
\usepackage{algpseudocode}
\usepackage{amsfonts}
\usepackage{amsmath}
\usepackage{amssymb}
\usepackage[ansinew]{inputenc}
\usepackage{xcolor}
\usepackage{mathtools}
\usepackage{graphicx}
\usepackage{caption}
\usepackage{subcaption}
\usepackage{import}
\usepackage{multirow}
\usepackage{cite}
\usepackage[export]{adjustbox}
\usepackage{breqn}
\usepackage{mathrsfs}
\usepackage{acronym}
%\usepackage[keeplastbox]{flushend}
\usepackage{setspace}
\usepackage{bm}
\usepackage{stackengine}

\usepackage{listings}

\lstset{%
 backgroundcolor=\color[gray]{.85},
 basicstyle=\small\ttfamily,
 breaklines = true,
 keywordstyle=\color{red!75},
 columns=fullflexible,
}%

\lstdefinelanguage{BibTeX}
  {keywords={%
      @article,@book,@collectedbook,@conference,@electronic,@ieeetranbstctl,%
      @inbook,@incollectedbook,@incollection,@injournal,@inproceedings,%
      @manual,@mastersthesis,@misc,@patent,@periodical,@phdthesis,@preamble,%
      @proceedings,@standard,@string,@techreport,@unpublished%
      },
   comment=[l][\itshape]{@comment},
   sensitive=false,
  }

\usepackage{listings}

% listings settings from classicthesis package by
% Andr\'{e} Miede
\lstset{language=[LaTeX]Tex,%C++,
    keywordstyle=\color{RoyalBlue},%\bfseries,
    basicstyle=\small\ttfamily,
    %identifierstyle=\color{NavyBlue},
    commentstyle=\color{Green}\ttfamily,
    stringstyle=\rmfamily,
    numbers=none,%left,%
    numberstyle=\scriptsize,%\tiny
    stepnumber=5,
    numbersep=8pt,
    showstringspaces=false,
    breaklines=true,
    frameround=ftff,
    frame=single
    %frame=L
}

\renewcommand{\thetable}{\arabic{table}}
\renewcommand{\thesubtable}{\alph{subtable}}

\DeclareMathOperator*{\argmin}{arg\,min}
\DeclareMathOperator*{\argmax}{arg\,max}

\def\delequal{\mathrel{\ensurestackMath{\stackon[1pt]{=}{\scriptscriptstyle\Delta}}}}

\graphicspath{{./figures/}}
\setlength{\belowcaptionskip}{0mm}
\setlength{\textfloatsep}{8pt}

\newcommand{\eq}[1]{Eq.~\eqref{#1}}
\newcommand{\fig}[1]{Fig.~\ref{#1}}
\newcommand{\tab}[1]{Tab.~\ref{#1}}
\newcommand{\secref}[1]{Section~\ref{#1}}

\newcommand\MR[1]{\textcolor{blue}{#1}}
\newcommand\red[1]{\textcolor{red}{#1}}
\newcommand{\mytexttilde}{{\raise.17ex\hbox{$\scriptstyle\mathtt{\sim}$}}}

%\renewcommand{\baselinestretch}{0.98}
% \renewcommand{\bottomfraction}{0.8}
% \setlength{\abovecaptionskip}{0pt}
\setlength{\columnsep}{0.2in}

% \IEEEoverridecommandlockouts\IEEEpubid{\makebox[\columnwidth]{PUT COPYRIGHT NOTICE HERE \hfill} \hspace{\columnsep}\makebox[\columnwidth]{ }}

\title{``We Rock the Hizzle and Stuff'' \\ hints on how to write a nice research essay}

\author{Michele Rossi$^\dag$, Author two$^\ddag$
\thanks{$^\dag$Department of Information Engineering, University of Padova, email: \{rossi\}@dei.unipd.it}
\thanks{$^\ddag$Author two affiliation, email: \{name.surname\}@dei.unipd.it}
\thanks{Special thanks / acknowledgement go here.}
}

\IEEEoverridecommandlockouts

\newcounter{remark}[section]
\newenvironment{remark}[1][]{\refstepcounter{remark}\par\medskip
   \textbf{Remark~\thesection.\theremark. #1} \rmfamily}{\medskip}

\begin{document}

\maketitle

\begin{abstract}
Future vehicular communication networks call for new solutions to support their capacity demands, by leveraging the potential of the \mbox{millimeter-wave} (\mbox{mm-wave}) spectrum. Mobility, in particular, poses severe challenges in their design, and as such shall be accounted for. A key question in \mbox{mm-wave} vehicular networks is how to optimize the \mbox{trade-off} between directive Data Transmission (DT) and directional Beam Training (BT), which enables it. In this paper, learning tools are investigated to optimize this \mbox{trade-off}. In the proposed scenario, a Base Station (BS) uses BT to establish a \mbox{mm-wave} directive link towards a Mobile User (MU) moving along a road. To control the BT/DT \mbox{trade-off}, a Partially Observable (PO) Markov Decision Process (MDP) is formulated, where the system state corresponds to the position of the MU within the road link. The goal is to maximize the number of bits delivered by the BS to the MU over the communication session, under a power constraint. The resulting optimal policies reveal that adaptive BT/DT procedures significantly outperform \mbox{common-sense} heuristic schemes, and that specific mobility features, such as user position estimates, can be effectively used to enhance the overall system performance and optimize the available system resources.\\

\MR{This is an example abstract. It is $204$ words long, I would say an abstract should not be longer than $250$ words and some Transactions journals of the IEEE are currently putting a strict limit of $200$ words. Here, you should briefly state:
\begin{enumerate}
\item the technical scenario/field of research and its timeliness/relevance in general (one sentence),
\item what you do in the report/paper and why it is important, how it advances the state of the art in its field (a few sentences),
\item summarize the main and best results of your study/proposal/method (one or two sentences),
\item (optional) how others could benefit from your results for further research, or within commercial products (one sentence).
\end{enumerate}
The abstract is one of the most important parts of the paper/report. You have roughly one minute to catch the reader's attention. A poor abstract may already move you towards the rejection side in the reviewer's decision process. In the abstract, 1) establish the context, 2) motivate the problem, 3) briefly describe the solution, and 4) present the main results of your work. Ideally, use one (short) sentence for each of the previously mentioned items to keep your abstract short. Overall, this should be a short summary of the whole content of your paper, including your results.}\\

\MR{See the abstract as a personal challenge for each of your papers. Finally, the abstract should contain the main message about your work, so that the reader will now what she/he can find even without reading it (as it is the case most of the times). The abstract is a mini-paper on its own and, as such, it is a major endeavor to write.}\\

\red{I suggest to write the Abstract as the very last thing. You may sketch it at the beginning, but then always finalize it at the end.}
\end{abstract}

\IEEEkeywords
Self Organizing Maps, Unsupervised Learning, Optimization, Neural Networks, Recurrent Neural Networks. \MR{A list of keywords defining the tools and the scenario. I would not go beyond {\it six} keywords.}
\endIEEEkeywords


% !TEX root = template.tex

\section{Introduction}
\label{sec:introduction}

Recognition of human daily activity is important for many
applications: from health care monitoring to security concerns, from
fitness tracking to user-adaptive systems.

With the wide spread of low-cost sensors on smartphones and
weareables, the development of mobile apps capable to track user
activities \textit{``in-the-wild''} leads to relevant challenges that
need to be tackle down. As stated in \cite{blunck2013heterogeneity}
many variables are involved both coming from users and
smartphones. Users are demographically different (age, stature,
weight, ...) and perform activities in different ways with their
style. Devices instead, share among them different operating-systems,
hardware and sensing capabilities.

In this paper we start from the model proposed in
\cite{ignatov2018real} and we show how the controlled environment
influences the classification.  In particular, no heterogeneity among
devices (and sensors) leads to biased data and not-natural settings.
For this reason we adopt the dataset provided in
\cite{stisen2015smart} which emphasizes devices heterogeneity.  We
further investigate the effectiveness of statistical and
manual-engineered features which are usually not enough in real
settings, due to the heterogeneous scenario. Moreover, we study the
effects of the orientation indipendent transformation as a
preprocessing data block, which should make data agnostic in terms of
sensor position and orientation.  With this work we aim to study
heterogeneity impairments in real use case scenarios using previous
state-of-the-art models, and then propose a novel learning framework
to tacke HAR impairments. Our framework will be made by a CNN
architecture augmented with features extracted from an autoencoder,
with an eye on mobile portability.  At the end we compare our results
with state-of-the-art works.

Our main contributions can be summarized as follows.

\begin{itemize}
  \item We propose an architecture that combines CNN and automatic
    feature extraction to perform HAR. In particular, augmenting a
    traditional CNN model with autoencoder's features rather than
    manual features can lead to better results.
  \item We study the importance of a good preprocessing pipeline, to
    mitigate heterogeneity when dealing with multiple types of
    smartphones. Also, orientation independent transformation can give
    promising results in real use case scenarios where smartphones can
    be in any position and orientation.
  \item We show that good results can be obtained
    w.r.t. state-of-the-art works also when few computational
    resources are available, i.e, smartphones.
\end{itemize}

This paper is structured as follows. In Section \ref{sec:related-work}
we describe the state-of-the-art, the system and data models are
respectively presented in Sections \ref{sec:processing-pipeline} and
\ref{sec:signals-and-features}. The proposed signal processing
technique is detailed in Section \ref{sec:learning-framework} and its
performance evaluation is carried out in Section
\ref{sec:results}. Concluding remarks are provided in Section
\ref{sec:concluding-remarks}.

%Conclusions:  So to mitigate all these problem when developing Human Activity Recognition (HAR) Systems a representative dataset of all the possibly end-users of out mobile app is required and other pre-processing technique are required for mitigating the heterogenity among devices.


% !TEX root = template.tex

\section{Related Work}
\label{sec:related-work}

HAR is a very wide discipline full of pitfalls and problems. A. Stisen et al. in \cite{stisen2015smart} studied the
technical details and problems due to heterogeneity among users,
sensors and environemnt. This heterogeneity, as they say, leads to
very low performances when models trained in controlled environment
are brought in real scenario.

One of this state-of-the-art model is presented in
\cite{ignatov2018real} where A. Ignativ uses a Convolutional Neural
Network (CNN) for local feature extraction together with simple statistical
features for preserving information about the global form of time
series. They also investigate the impact of time series length on the
recognition accuracy. The accuracy of the proposed approach is
evaluated on two commonly used WISDM and UCI datasets that contain
labeled accelerometer data from 36 and 30 users respectively, and in
cross-dataset experiment. The results show that the proposed model
demonstrates state-of-the-art performance while requiring low
computational cost and no manual feature engineering as for the example in work TODO where the authors used 150 manual extracted features.

In this paper we start from this model and we show how the controlled
environemnt influences the classification. In particular, no
heterogeneity among devices (and sensors) leads to biased data and
not-natural settings.  For this reason we adopt the dataset provided
in \cite{stisen2015smart} which emphasizes devices heterogeneity.
Furthermore, as the main goal is HAR ``in the wild'', we introduce the
Orientation Indipendent Transformation presented by M. Gadaleta and
M. Rossi in \cite{gadaleta2018idnet} as a solution for the problem of
data gathered by sensors in any position and orientation. We further investigate the
effectiveness of statistical/manual features which are usually not
enough in real settings, due to the heterogeneous scenario. For this
reason we introduce a technique described by F. Gu in
\cite{gu2018locomotion} which employs an autoencoder to atuomatically
extract robust features.


% !TEX root = template.tex

\section{Processing Pipeline}
\label{sec:processing_architecture}

\begin{figure*}[h]
	\centering
	\includegraphics[width=1\textwidth]{images/processing_pipeline.jpg}
	\caption{Processing pipeline}
\end{figure*}

The task of HAR 'in the wild' is a difficult problem, and many aspects need to be considered when dealing in that situation. As reported in \cite{blunck2013heterogeneity} the three major types of heterogeneties which yield impairments in Human Activity recognition are:
\begin{itemize}
	\item \textbf{Sensor Biases}: To keep the overall cost of a smartphone low, low costs accelerometer and gyroscope sensor are used, yielding a poor-calibrated, inaccurate an of limited granularity and range acquired signals. So, among this type of sensors we could observe differences in precision, resolution, range and also biases. Usually an initial sensors calibration are made by smartphones manufactures, but due to rotation or misalignment of the sensor to the circuit board of the final product, this could introduce errors. Furthermore, if a device experience shock, e.g falling on the ground, the sensor can be misaligned causing unwanted biases.
	\item \textbf{Sampling Rate Heterogeneity}: Often popular smartphones vary in terms of the default and supported sampling frequencies for accelerometer and gyroscope sensor. In the dataset TODO riportare link used for this experiment for example we are dealing with smartphone where the sampling frequency varies from 50Hz to 200Hz.
	\item \textbf{Sampling Rate Instability}: This phenomenon is specific to a single device and regards the regularity of the time span between successive measurements. Different factors could accentuate this problem, including heavy multitasking or high I/O load in the mobile device. Multitasking effect in particular is a major problem: smartphone usually prioritizes among various running tasks and doing so could extremely affects the sensor sampling of and HAR application running on the device. In our collected dataset with a 100 Hz sampling rate, we observe a time span between consecutive measurement of TODO, even if the smartphone was hold in \textit{airplane mode} to reduce at the minimum this effect. The Fig. TODO shows the amount of different time-span present in our dataset.
\end{itemize}

Furthermore, if we considered a real use case scenarios of a HAR mobile app we must also consider that smartphones can be positioned and oriented in different ways. For example a smartphone could lay in trouser pockets (back and front) or maybe inside accessories like in a pouch or in a bag in different orientation. These different initial model settings have huge effects in prediction accuracy of an HAR predictor, especially if the model has been trained on a dataset that consist of activity measurement coming from only one fixed position and orientation of the smartphone, as is usually the case. 

To tackle all these problems we decide to adopt a smart pre-processing pipeline as show in Fig. TODO, consisting of 3 main blocks.

The first block called Linear Interpolator is in charge of mitigate the problems regarding the Sampling Rate Heterogeneity and Sampling Rate Instabilities as discussed previously. It main purpose is to down-sample the input data to a fixed sampling rate in our case 50 samples/second. 

The second block called Orientation Independent Transformation is used to represent data coming form different orientation of the smartphone in a rotation independent space. In this way all the signal are projected in a new space whose orientation is independent of that of the smartphone and aligned with gravity and the direction of motion. In this way an user can place his smartphone in whatever position he or she wants, without any impairment in the performance of the prediction model.

Our last block consists of a data centering operation, applied for centering signals among y-axis that are presented only for the Covolutional Layer and not for the Autoencoder. The reason of this choice would be clear in the section TODO. As reported in \cite{ignatov2018real}, time series centering standardize the input data, making the task for the CNN easier. Data normalization instead must be avoided because does not help in this situation since it significantly distorts time series shape, removing magnitude information which is critical for activities differentiation.

After the preprocessing pipeline we adopt a novel Learning framework. It is composed of CNN augmented with features coming from an autoencoder. As discussed in \cite{ignatov2018real} CNNs learns filters that are applied to small sub-regions of the data, and therefore they are able to capture local data pattern and their variations. Additionally, due to a small number of connections and high parallelism the amount of computations and running time of CNNs is significantly lower compared to other deep learning algorithm. This yield these model perfect for real-time HAR apps, where these models could also run in a restrict environment as one like smartphones where computation resources are limited. The only drawback of CCNs is that they fall behind in capturing global properties of the signal, and as proposed in \cite{ignatov2018real} they eliminate this problem by augmenting CNNs with some basic statistical features that comprise this aspects of the data. But as opposed of what done in this latter work, where they used manual extracted features, we decided to opt for a autoencoder features extractors which can provide more robust feature. For this reason we train an auto-encoder separately on the training data and then use the encoder part to augment the CNN features used for the last classification Feed forward Neural Network. 

\section{Signals and Features}
\label{sec:model}

In questo caso si va nel dettaglio della parte di preprocessing (parte trattegiata nel mio schema come preprocessing)

Parlare del dataset, come sono stati processati i dati (interpolazione lineare), parlare della time-window applicata di 2,5s, e parlare del rotational indipendent trasform per ruorare nuovamente i dati. L'applicazione del centering per la sola rete CNN e l'estrazione delle basic feature per risimulare il lavoro fatto in \cite{ignatov2018real}.

\subsection{Dataset \& Meausurement Setup}
Parlare di come abbiamo splittato il dataset, quindi escludendo gli utenti a e b per fare in modo di testare le performance cambiando compltamente utente.

Introduzione e spiegazione del dataset eterogeneo, preso da: riportare link.

Accenare dell'acquisizione di un nostro dataset alla stessa maniera, a 100 HZ in varie posizioni per testare poi anche tutto il dataset!

\subsection{Signal preprocessing}

\begin{itemize}
	\item interpolazione lineare / downsampling
	\item rotational invariant citare tutti i paper e come funziona
	\item centering
\end{itemize}

\subsection{Feature vector}
\begin{itemize}
	\item window strategy
	\item basic feature extraction
	\item autoencoder feature extraction
\end{itemize}

\section{Learning Framework}
\label{sec:learning_framework}

\begin{figure*}[h]
	\centering
	\includegraphics[width=1\textwidth]{images/full_architecture.jpg}
	\caption{Learning Framework}
\end{figure*}

Desctivere qui invece la parte tratteggiata come learning framework

Descivere prima l'autoencoder, come è stato costruito e come viene allenato. TODO luca

Passare alla decrizione della mia archittetura, come sono stai selezionati gli iper-parametri, ecc ecc.



% !TEX root = template.tex

\begin{table*}[t]
	\begin{center}
		\begin{tabular}{ p{7cm}p{2cm}p{2cm}p{2cm}p{2cm} }
			\hline
			Method & Accuracy & Precision & Recall & F1-Score \\
			\hline
			CNN + No centering & 77.0 & 78.0 & 75.8 & 76.8 \\
			CNN + No centering + Manual F. & 75.6 & 76.8 & 73.9 & 75.3 \\
			CNN + Centering + Manual F. & 86.2 & 86.9 & 70.2 & 77.6 \\
			\textbf{CNN + Centering + Encoder F.} & \textbf{89.2} & \textbf{88.9} &  \textbf{86.1} & \textbf{86.1} \\
			\hline
		\end{tabular}
		\caption{\label{tab:model-performance} HD classification results with different featueres CNN augumentation and data preprocessing. The tests are made with our CNN with OIT, Data centering, 196 (1x16) filters, max-pool (1x4), 64 fully connected neurons, l2-regularization of $5e^{-5}$ and Adam learning rate of $2e^{-5}$.}
	\end{center}
\end{table*}

\section{Results}
\label{sec:results}

\subsection{Heterogenity Dataset (HD)}
\label{subsec:heterogeneity-dataset}

In this section we evaluate performances between previous works and
different model architectures and the influences of the
preprocessing techniques applied here. Similar to what was done in
\cite{ignatov2018real}, we carried out from the HD dataset some
representative users to test the model and then use the remaining ones
to train the model. In this case we selected users \textit{a} and
\textit{b} since we found that they are very representative among all
others. In fact models tend to be less precise with user \textit{a}
and more accurate with user \textit{b}. This mainly depends on the
user's style in walking, doing stairs and so on. In this way we are
able to compare results with other works that usually tend to evaluate
performances on unseen user.

In this training and test set settings, we evaluate the best
hyper-parameters for the proposed architecture model excluding OIT
preprocessing block, since this dataset was collected with a fixed
orientation. The rest of preprocessing techniques are enabled, if not
specified.

\textbf{Autoencoder.} In order to search for best autoencoder model
hyper-parameters we performed a grid search. Results are reported in
Tab. \ref{tab:ae-hyperparams} where values that performs well on the
validation dataset are reported in bold. The best hyper-parameters
model used to be the one with a relatively small code size: from 24 to
36 features which is also a good thing as we do not want the
autoencoder to learn the identity, but only to keep useful
information. In this settings we get an MSE of $0.871255$.
\begin{table}[h]
  \centering
  \begin{tabular}{lp{4cm}}
    \hline
    Hyperparameter & Values \\
    \hline
    code size & \{2, 3, 4, 5, 6, 12, 18, 24, 30, \textbf{36}, 42, 48, 54, 60, 72\} \\
    batch size & \{32, \textbf{128}\} \\
    epochs & \{\textbf{150}, 200\} \\
    \hline
  \end{tabular}
  \caption{Grid-search for best hyper-parameters on autoencoder}
  \label{tab:ae-hyperparams}
\end{table}

Also, for checking autoencoder results we fine-tuned KNN parameters
with a grid search, looking for best values of distance measure and
number of neighbors. The Tab. \ref{tab:knn-grid-search} show the values.
We select the euclidean distance measure and $5$ as number of neighbors.
Due to time reasons we do not investigate further the FFNN model instead.
The obtained performances are quite satisfying: we get $76.2$\% accuracy
for KNN and $81.8$\% for FFNN.
\begin{table}[h]
  \centering
  \begin{tabular}{p{2cm}p{4.5cm}}
    \hline
    Hyper-parameters & Values \\
    \hline
    Distance measure & \{\textbf{euclidean}, manhattan, chebyshev, minkowski, standardized euclidean, mahalanobis\} \\
    Number of neighbors & \{3, 4, \textbf{5}, 6, 7, 8\} \\
    \hline
  \end{tabular}
  \caption{Grid-search for KNN classifier}
  \label{tab:knn-grid-search}
\end{table}

From the grid-search on KNN we surprisingly noticed that performances
do not vary significantly among different hyper-parameters: we get from
$76.7$\% to $80.9$\% accuracy. This may be an indication of the
maximum capability of this autoencoder model, so to obtain better
performances we have to add complexity to the model.

\textbf{CNN Network.} We fist test how number of convolutional filters in CL1 and number of dense neurons in FL2 will influence classification performances. The results obtained are presented in Tab. \ref{tab:model-selection}. We decided to choose 196 convolutional filters and 64 dense neurons thanks to its balance between accuracy and F1-Score performances, obtaining $88.9\%$ and $81.6\%$ respectively. To compare our result with that obtained in the original work \cite{stisen2015smart}, we decided to perform their \textit{Leave-one-user-out cross validation} evaluation, consisting of test the model with data from one user, and train with data from all the others in a cross validation fashion and then averaging the metrics obtained. In this evaluation setting we obtained an average F1-score of $85.8\%$, beating their best model result of nearly $10\%$ more in F1-Score metric. This prove also that using users a and b to do our evaluation is a good compromise of the real \textit{Leave-one-user-out cross validation} evaluation performances, since we obtain nearly the same results ($90.2\%$ instead of $88.9\%$ in accuracy and $85.8\%$ instead of $81.6\%$).

\begin{table}[h]
	\begin{center}
		\begin{tabular}{ p{1.8cm}p{1.7cm}p{1.7cm}p{1.7cm} }
			\hline
			CNN Filters & Dense Neurons & Accuracy & F1-Score \\
			\hline
			196 & 1024 & 84.6 & 75.3 \\
			196 & 512 & 86.1 & 75.7 \\
			\textbf{196} & \textbf{64} & \textbf{88.9} & \textbf{81.6} \\
			96 & 1024 & 82.8 & 69.9 \\
			96 & 512 & 84.4 & 73.7 \\
			96 & 64 & 89.0 & 73.0 \\
			48 & 1024 & 80.0 & 79.4 \\
			48 & 512 & 83.7 & 74.9 \\
			48 & 64 & 84.0 & 72.3 \\
			\hline
		\end{tabular}
		\caption{\label{tab:model-selection} HD classification results with data centering and manual features augmented CNN}
	\end{center}
\end{table}


To better appreciate how our preprocessing blocks affects model overall performances, we also try to disable or enable some of them. In Tab. \ref{tab:model-performance} we report our obtained results experiments. We see that augmenting the CNN with manual extracted feature when data are not centered, lead to no significant change in performances, instead when also enabling data centering preprocessing with manual features augmented CNN the model obtain nearly $10\%$ more in accuracy and precision metrics. This prove the benefits of data centering stated previously. However we were not able to reach the same performances presented in \cite{ignatov2018real}, where the authors obtained in the same exact settings an accuracy of $97.6\%$. These empirically confirm that performances of state-of-the-art models trained with one type of sensors are worse when dealing with smartphone sensors heterogeneity. Moving on, augmenting the CNN with encoder feature lead the model to better performances, meaning that encoder features are more robust that manual features, as explained previously. A confusion matrix in this latter setting is reported in Fig. \ref{fig:cnn-confusion-matrix} where we could see that the model performs nicely overall in all the considered activities, with some difficulties in distinguishing between stationary activities, stand and sit, and walk with stairs activities since probably they are very similar in sensor signals foot-prints.

\begin{figure}[h]
	\centering
	\includegraphics[width=0.5\textwidth]{images/confusion_matrix.png}
	\caption{CNN confusion matrix}
	\label{fig:cnn-confusion-matrix}
\end{figure}


\subsection{Oriented Dataset (OD)}

With these new collected dataset we want to test our model
performances in a real use case scenario, where smartphone could be
placed in different positions and orientations. In this case we
trained the model with the entire HD as training set, and then use the
OD as validation set. It is important to remember that in this case we condensed the sit and stand activities of HD into a one class no activity category, since OIT makes these two stationary activities indistinguishable from each other.

\textbf{Autoencoder.}  An interesting result, as the
Tab.~\ref{tab:ae-loss} confirms, is that OIT is a necessary operation
when dealing with HAR signals. For example, without OIT we can see
that the autoencoder trained and tested on same data goes from
$0.747241$ to $10.421337$ MSE: more than $10$x worse. Furthermore,
data from hand or pocket-up/down do not inflate the loss too
much. This is good because it indicates that autoencoder is producing
robust features.

\begin{table}[h]
  \centering
  \begin{tabular}{lr}
    \hline
    Scenario & Loss (MSE) \\
    \hline
    HD + OIT + OD validation & 0.747241 \\
    HD + OIT + OD validation (allpos) & 0.821241 \\
    HD + OD validation & 10.421337 \\
    \hline
  \end{tabular}
  \caption{Autoencoder loss on different scenarios}
  \label{tab:ae-loss}
\end{table}

The Tab. \ref{tab:knn-metrics} and \ref{tab:ffnn-metrics} show
respectively KNN and FFNN evaluation for the best autoencoder with
$36$ features and the two classifier described in
Sec. \ref{subsec:autoencoder} with hyper-parameters selected in
Sec. \ref{subsec:heterogeneity-dataset}. From this results, we can
observe that KNN is more stable and not influenced by sensor's
position/orientation w.r.t. FFNN. Also, the two models preserve order:
evaluation on \textit{pouch} position gets best results on both
models, while the worse position is \textit{hand+pocket}, as we
expected.

\begin{table}[h]
  \centering
  \begin{tabular}{lrrrr}
    \hline
    Positions & Accuracy & Precision & Recall & F1-score \\
    \hline
    Pouch & 80.1 & 86.4 & 80.7 & 79.0 \\
    Hand+Pocket & 79.5 & 83.7 & 79.5 & 79.6 \\
    All & 80.0 & 83.8 & 79.1 & 78.2 \\
    \hline
  \end{tabular}
  \caption{KNN evaluation onto OD between smartphone positions (pouch
    left/right/top/back, hand and pocket-up/down, all positions)}
  \label{tab:knn-metrics}
\end{table}

\begin{table}[h]
  \centering
  \begin{tabular}{lrrrr}
    \hline
    Positions & Accuracy & Precision & Recall & F1-score \\
    \hline
    Pouch & 74.4 & 84.7 & 74.4 & 74.8 \\
    Hand+Pocket & 67.8 & 78.0 & 67.8 & 70.0 \\
    All & 69.9 & 81.3 & 69.9 & 72.4 \\
    \hline
  \end{tabular}
  \caption{FFNN evaluation onto OD between smartphone positions (pouch
    left/right/top/back, hand and pocket-up/down, all positions)}
  \label{tab:ffnn-metrics}
\end{table}

\textbf{CNN Network.}  As reported in
Tab. \ref{tab:model-oit-performance}, we managed to get great results
also on data gathered with new sensor's position and orientation. We
noticed a performance drop of nearly $15\%$ for \textit{hand+pocket}
data, but this could be due to the CNN architecture which focus more
on the local shape and less on the real information. However, we
expected this behavior. 

\begin{table}[h]
  \begin{center}
    \begin{tabular}{p{1.8cm}rrrr}
      \hline
      Positions & Accuracy & Precision & Recall & F1-score \\
      \hline
      Pouch & 85.3 & 92.0 & 73.8 & 81.9 \\
      Hand+Pocket & 70.5 & 79.5 & 63.0 & 70.2 \\
      All & 78.0 & 84.0 & 69.0 & 75.7 \\
      \hline
    \end{tabular}
    \caption{CNN classification comparisons onto \textit{OD} between
      smartphone positions (pouch left/right/top/back, hand and
      pocket-up/down and all positions).}
    \label{tab:model-oit-performance}
  \end{center}
\end{table}

In conclusion, OIT revealed to be an extremely useful preprocessing
technique both for autoencoder and CNN. Without OIT, the drop for the
CNN was of nearly $45$\% for all metrics!  As Tab. confirms, combining
the power of the CNN with automatically extracted features and smart
data preprocessing could lead to good results even with new problem
settings.

\begin{table}[h]
  \begin{center}
    \begin{tabular}{lr}
      \hline
      Classifier & Accuracy \\
      \hline
      CNN + AE features & 85.3 \\
      KNN & 80.1 \\
      FFNN & 74.4 \\
      \hline
    \end{tabular}
    \caption{Comparison between KNN, FFNN and CNN classifiers onto \textit{pouch} data from OD.}
    \label{tab:classifiers-comparison}
  \end{center}
\end{table}


% !TEX root = template.tex

\section{Concluding Remarks}
\label{sec:concluding-remarks}

In this paper we proposed our solution to perform HAR 'in the wild', that could be implemented in real case scenario. Dealing with major type of heterogeneity and real use case scenario were smartphone could be positioned and oriented in any way, can be a difficult task. Using a good pre-processing pipeline, implementing linear interpolation, orientation independent transformation and Data Centering we were able to mitigate HAR impairments due to latter problems. To perform HAR, various techniques were proposed in literature: one of them very promising is the use of CNN to exploit automatic feature extraction and then compute the final classification. Augmenting CNN with more robust features like the one coming from the encoder part of an autoencoder, we were able to outperform previous results in the original work, which uses the Heterogeneity Dataset. Moreover we demonstrate that orientation independent transform is essential, allowing us to use models trained with dataset made in controlled environments, in real use case scenario with promising results. 

In future works it could be usefull to test the performance of this model with a more challenging dataset that includes: various types of persons, ie with different weights, heights, ethnicity, etc. and various type of routes where activities are performed. Moreover it could be useful to test it with subjects wearing different types of shoes and clothes. We would like also to extend the recognizable activities and also includes places were a user could be, ie be on a car, on a bus, on a train, on a plane etc. Another useful contribution, that due to lack of time we weren't able to test, is the use of open set classification techniques in which the system should reject unknown/unseen activities. Using a smartphone app, where our model was implemented, we saw that the model is constantly trying to predict the learned activities even if the user is completely doing other things with his smartphone.

With this work we learn that datasets are very important when dealing with machine learning / deep learning applications: a huge and good representative dataset of the final use case scenario onto which the model will be used, is extremely important. Moreover, pre-processing technique are very powerful and finding the right one for the project could lead to completely different results, without changing anything of the final model architecture. Furthermore, Autoencoder feature extraction could be very powerfull to augment other popular deep learning techniques.

During our project work we encounter also many problems, one of them was the bad optimization NVIDIA drivers for Ubuntu OS. The machine onto which we were performing our tests, very often crashes due to bad GPU's memory management. Also we encounter a very annoying bug using the prefetching Tensorflow dataset feature, which provide to us very different model metrics results when training multiple times the model with the same hyper parameters. After disabling that option the system provide to us comparable results when dealing with same hyper-parameters. Also we have to report that some of the paper that we red for our project were poorly written, in particular autors were not clear to present their proposed architectures. Sometimes they didn't make explicit model hyper-parameters and training settings, making it difficult to replicate their work.


\section{Exam rules}

What you need to do to pass the exam:
\begin{itemize}
\item Optional: team up with another student. Max. group size is \textbf{two students} per group;
\item Identify a project to work on, devise your own neural network architecture and test it on the provided dataset;
\item \textbf{Prepare a written project report} including: i) diagrams, ii) configuration pars, iii) results, iv) your discussion;
\item \textbf{Prior to presenting your work}: upload i) your written report and ii) the code;
\item \textbf{Present your work} using slides (max. duration is 20 minutes): take turns in presenting your work, your individual contribution to the project should clearly emerge. Optional: a final and quick demo with running code is appreciated and will be considered in the calculation of the final grade (see below).
\end{itemize}

Your final grade will be obtained taking into account the following criteria:
\begin{itemize}
\item \textbf{Project} (60 points): originality (10 pt.) - data preprocessing techniques (10 pt.) - learning architectures (20 pt.) - comparison against other/existing approaches (10 pt.) - live demo of the code (10 pt.)
\item \textbf{Written report} (40 points): clarity of exposition (10 pt.) - completeness (10 pt.) - analysis of results (number and type of metrics used) (20 pt.)
\item \textbf{Oral exposition} (20 points): duration (your talk must take max. 20 minutes, using slides) (10 pt.) - clarity of exposition (10 pt.)
\end{itemize}

The final grade will be computed as
\begin{equation}
\textrm{grade} = \frac{\textrm{tot\_points} \times 30}{110}
\end{equation}

\bibliography{biblio}
\bibliographystyle{ieeetr}

\end{document}


