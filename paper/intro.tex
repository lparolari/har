% !TEX root = template.tex

\section{Introduction}
\label{sec:introduction}
Recognition of the activity that a user is currently doing is important not only for health care monitoring or security concerns, but also for developing mobile apps that are able to use this information to improve our day life. \\

Developing mobile apps that are capable of tracking user activities that are used "in the wild" contexts, leads important challenges that need to be tackle down. As stated in \cite{blunck2013heterogeneity} many variables are involved both coming from users and smartphones. Users are demographically different (age, stature, weight, ...) and perform activities in different ways, using their devices in their own ways. Devices instead share among them different operating-systems, hardware and sensing capabilities.

Different solution were proposed in literature to perform Human Activity Recognition (HAR)[citare tutti i paper] capable of reaching high performance, but when they are used in real scenarios the performance usually decrease a lot.

TODO: Citare inoltre il paper \cite{chen2020deep} recente che dice parecchie cose sugli algoritmi di learning attuali e tutti i problemi che ci sono in questo caso 



%Conclusions:  So to mitigate all these problem when developing Human Activity Recognition (HAR) Systems a representative dataset of all the possibly end-users of out mobile app is required and other pre-processing technique are required for mitigating the heterogenity among devices.

