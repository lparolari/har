% !TEX root = template.tex

\section{Introduction}
\label{sec:introduction}

Recognition of human daily activity is important for many
applications: from health care monitoring to security concerns, from
fitness tracking to user-adaptive systems.

With the wide spread of low-cost sensors on smartphones and
weareables, the development of mobile apps capable to track user
activities \textit{``in-the-wild''} leads to relevant challenges that
need to be tackle down. As stated in \cite{blunck2013heterogeneity}
many variables are involved both coming from users and
smartphones. Users are demographically different (age, stature,
weight, ...) and perform activities in different ways with their
style. Devices instead, share among them different operating-systems,
hardware and sensing capabilities.

In this paper we start from the model proposed in
\cite{ignatov2018real} and we show how the controlled environment
influences the classification.  In particular, no heterogeneity among
devices (and sensors) leads to biased data and not-natural settings.
For this reason we adopt the dataset provided in
\cite{stisen2015smart} which emphasizes devices heterogeneity.  We
further investigate the effectiveness of statistical and
manual-engineered features which are usually not enough in real
settings, due to the heterogeneous scenario. Moreover, we study the
effects of the orientation indipendent transformation as a
preprocessing data block, which should make data agnostic in terms of
sensor position and orientation.  With this work we aim to study
heterogeneity impairments in real use case scenarios using previous
state-of-the-art models, and then propose a novel learning framework
to tacke HAR impairments. Our framework will be made by a CNN
architecture augmented with features extracted from an autoencoder,
with an eye on mobile portability.  At the end we compare our results
with state-of-the-art works.

Our main contributions can be summarized as follows.

\begin{itemize}
  \item We propose an architecture that combines CNN and automatic
    feature extraction to perform HAR. In particular, augmenting a
    traditional CNN model with autoencoder's features rather than
    manual features can lead to better results.
  \item We study the importance of a good preprocessing pipeline, to
    mitigate heterogeneity when dealing with multiple types of
    smartphones. Also, orientation independent transformation can give
    promising results in real use case scenarios where smartphones can
    be in any position and orientation.
  \item We show that good results can be obtained
    w.r.t. state-of-the-art works also when few computational
    resources are available, i.e, smartphones.
\end{itemize}

This paper is structured as follows. In Section \ref{sec:related-work}
we describe the state-of-the-art, the system and data models are
respectively presented in Sections \ref{sec:processing-pipeline} and
\ref{sec:signals-and-features}. The proposed signal processing
technique is detailed in Section \ref{sec:learning-framework} and its
performance evaluation is carried out in Section
\ref{sec:results}. Concluding remarks are provided in Section
\ref{sec:concluding-remarks}.

%Conclusions:  So to mitigate all these problem when developing Human Activity Recognition (HAR) Systems a representative dataset of all the possibly end-users of out mobile app is required and other pre-processing technique are required for mitigating the heterogenity among devices.
