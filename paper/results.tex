% !TEX root = template.tex

\section{Results}
\label{sec:results}

\begin{center}
	\begin{tabular}{ p{2cm}p{2cm}p{2cm} } 
		\hline
		cell1 & cell2 & cell3 \\ 
		\hline
		cell4 & cell5 & cell6 \\ 
		cell7 & cell8 & cell9 \\ 
		\hline
	\end{tabular}
\end{center}

Io direi che in questa parte partiamo con il dataset eterogeneo, facendo vedere come si comporta anche con le posizione sit e stand, e dopodichè passando ai risultati ottenuti con il rotational indipendent far vedere quello che abbiamo ottenuto nel dataset nostro, dove le attivita' sit e stand sono state compressate in no\_activity!

\subsection{Original settings}

Parlare dei risultati ottenuti da luca con un semplice autoencoder sia K-NN classifier che con FFNN alla fine. Un concetto alla volta!

Passare alla mia archittetura e spiegare come sono stati scelti i paramentri, poi far vedere che togliende le basic feature il modello decrementa la sua accuracy, e aggiungendo invece quelle de''autoencode il modello migliora. 

Calcolare anche la F1 score e prbabilmente abbiamo migliorato di molto il modello proposto in \cite{blunck2013heterogeneity}

\subsection{Advance settings}
Passare al nostro dataset e far vedere l'importanza del rotational invariant!!! Che migliora sensibilmente le performance ovviamente dato che la rete è allenata in posizione fisse.

RIPORTARE TUTTE LE METRICHE, anche confusion matrix e argomentarle
